\documentclass[11pt]{article}

\usepackage{fullpage, xspace, hyperref, amsmath, multirow}

\def\qcmlver{0.1\xspace}
\def\qcml{\texttt{QCML}\xspace}
\def\cvxpy{\texttt{CVXPY}\xspace}
\def\cvx{\texttt{CVX}\xspace}
\def\cvxgen{\texttt{CVXGEN}\xspace}

% for captions
\makeatletter
\long\def\@makecaption#1#2{
   \vskip 9pt 
   \begin{small}
   \setbox\@tempboxa\hbox{{\bf #1:} #2}
   \ifdim \wd\@tempboxa > 5.5in
        \begin{center}
        \begin{minipage}[t]{5.5in}
        \addtolength{\baselineskip}{-0.95pt}
        {\bf #1:} #2 \par
        \addtolength{\baselineskip}{0.95pt}
        \end{minipage}
        \end{center}
   \else 
	\hbox to\hsize{\hfil\box\@tempboxa\hfil}  
   \fi
   \end{small}\par
}
\makeatother

% text abbrevs
\newcommand{\eg}{{\it e.g.}}
\newcommand{\ie}{{\it i.e.}}
\newcommand{\etc}{{\it etc.}}

% math defs
\newcommand{\reals}{{\mbox{\bf R}}}
\newcommand{\dom}{\mathop{\bf dom}}


\title{\qcml version \qcmlver \\
A Python Toolbox for Matrix Stuffing}

\author{
Eric Chu\\\texttt{echu508@stanford.edu}
\and
Stephen Boyd\\\texttt{boyd@stanford.edu}
}

\date{\today}

\begin{document}
  \maketitle
  
\qcml is a Python module that provides a simple and lightweight domain
specific language (DSL) for describing convex optimization problem families
representable as second-order cone problems (SOCP). It is a code generator
that produces lightweight Python, Matlab, or C code for \emph{matrix
stuffing}. This allows one to extend SOCP solvers in these langauges to a 
variety of problem classes. It is intended to complement the paper, 
\emph{Code Generation for Embedded Second-order Cone Programming} \cite{CPD:13}, 
although the main ideas in its
implementation are well-documented in \emph{Graph Implementation for Nonsmooth Convex Programs}
\cite{GB:08}.

\qcml comes with a small example library and consists of a single Python
object, \texttt{QCML}, with three methods
\begin{itemize}
\item \texttt{parse}, which parses an optimization problem specified as a
string and checks that it is convex,
\item \texttt{canonicalize}, which symbolically converts the problem into
a second-order cone program,~and
\item \texttt{codegen}, which produces source code that performs the
transformation from problem-specific data to second-order cone problem data
and back.
\end{itemize}
The \texttt{codegen} method can be used to generate Python, C, and Matlab
source code.

Some important features:
\begin{itemize}
\item \qcml is similar in spirit to \cvx, \cvxgen, and 
\cvxpy. Unlike these packages, however, \qcml exposes and generates code
for matrix stuffing, allowing the resulting code to be paired with any solver
that targets a standard SOCP.
\item \qcml is capable of generating fully abstract code with problem data and
dimensions as input to the \emph{generated} code, although it can specialize
the code if given problem dimensions before code is generated.
\item \qcml supports matrix parameters with arbitrary sparsity patterns.
\end{itemize}
That said, there are several important caveats:
\begin{itemize}
\item \qcml is intended for use by \emph{developers} wishing to employ convex
optimization in their source code with little to no overhead from the modeling 
layer. Typically, these are users familiar with \cvx or \cvxpy who wish to
do away with modeling overhead in their applications.
\item \qcml is not a full-featured modeling language. Several features are
conspicuously missing and unless specifically requested, will remain missing.
\item For general purpose convex optimization in Python with more flexibility
than \qcml, we recommend \cvxpy 
(\url{http://github.com/cvxgrp/cvxpy}).
\item For general purpose convex optimization in Matlab, we recommend 
\cvx (\url{http://cvxr.com}).
\end{itemize}
Please send your feedback or report any bugs through our Github repository.

The rest of this document covers the basic use of \qcml and walks through a
full example with a lasso ($\ell_1$-regularized least-squares) problem. 
\qcml is licensed under BSD.

\newpage
\tableofcontents
\newpage

\section{Installing \qcml}
\qcml depends on the following:
\begin{itemize} 
\item Python 2.7.2+ (no Python 3 support)
\item \href{http://www.dabeaz.com/ply/}{\tt PLY}, the Python Lex-Yacc parsing framework,
  available as the {\tt python-ply} or {\tt py-ply} package in most distributions.
\item {\tt ECOS} (\url{http://github.com/ifa-ethz/ecos})
\item {\tt NUMPY} (\url{http://numpy.org})
\item {\tt SCIPY} (\url{http://scipy.org})
\end{itemize}
For (some) unit testing, we use Nose (\url{http://nose.readthedocs.org}).

Once these dependencies are installed, typically through a package manager,
\qcml can be installed by typing
{\tt
\begin{tabbing}
  \qquad \= \$ python setup.py install
\end{tabbing}
}
\noindent at the command line. The unit tests can be run with {\tt nosetests}.

\section{\qcml language}
Optimization problems in \qcml are specified as \emph{strings} which must adhere to 
the rules of the \qcml modeling language. We describe that language and its
rules here.

\subsection{Types}
There are three basic types in \qcml:
\begin{itemize}
\item {\tt dimension} (or {\tt dimensions} for multiple dimension declarations)
\item {\tt parameter} (or {\tt parameters} for multiple parameter declrations)
\item {\tt variable} (or {\tt variables} for multiple variable declarations)
\end{itemize}
Objects named and declared in this fashion are entirely \emph{abstract}.
Numeric literals such as {\tt 2}, {\tt 0.52}, \etc in \qcml are
\emph{concrete} and can be used directly in the \qcml problem string. 

The {\tt dimension} objects are used to specify the (optional) sizes of {\tt
variable}s and {\tt parameter}s. So a single {\tt variable} or {\tt
parameter} object can represent an collection (or array) of scalar variables
or parameters with abstract dimension. (For more detail on abstract
dimensions, consult \S\ref{s-abstract-dims}.) At the moment, {\tt variables}
can represent vectors (one-dimensional arrays), and {\tt parameters} can
represent vectors or matrices (two-dimensional arrays). If a {\tt parameter}
represents a matrix, it is assumed to represent a \emph{sparse} matrix.

A {\tt parameter} may optionally take a sign modifier ({\tt positive} or 
{\tt negative}). These are shorthand for `nonnegative' and `nonpositve', 
respectively.

To declare an object, give its type, followed by its name, and (optionally) 
its size: for instance,
{\tt
\begin{tabbing}
  \qquad \= variable x(n)
\end{tabbing}}
\noindent declares a vector variable named {\tt
x} with length {\tt n}. 
Integer literals can also be used in place of dimensions. A more involved
example is the following code
{\tt
\begin{tabbing}
  \qquad \= dimensions m n \\
  \> variables x(n) y(m) z(5)\\
  \> parameter A(m,n) positive \\
  \> parameters b c(n)
\end{tabbing}}
\noindent which declares two dimensions, {\tt m} and {\tt n}; three variables
{\tt x}, {\tt y}, and {\tt z} of length {\tt n}, {\tt m}, and {\tt 5},
respectively; an elementwise positive (sparse) parameter matrix, {\tt A}; the
scalar parameter {\tt b}; and the vector parameter {\tt c} with dimension
{\tt n}.

\subsection{Operators}
\qcml provides a set of operators for use with modeling. These are used in
conjunction with the basic {\tt variable} and {\tt parameter} types to
construct (affine) expressions. The standard (infix) operators are: {\tt +},
{\tt -}, {\tt *}, and {\tt /}. The division operator {\tt /} has the
restriction that both operands must be numeric constants. The {\tt -}
operator can also be used as a prefix to denote negation.

\subsection{Functions}
\qcml also provides a set of atoms (or functions) which take expressions constructed from the basic operations as arguments. These functions have additional properties associated with them:
\begin{itemize}
\item the \emph{sign} of the output expression (either positive, negative, or unknown),
\item the \emph{monotonicity} of the function in each argument 
(either increasing, decreasing, or nonmonotone),
\item and the \emph{curvature} of the function (either convex, concave, or affine).
\end{itemize}
Roughly speaking, these can be associated with the sign of the zeroth, first,
and second derivatives of the function, respectively. (Although not all
functions we provide are differentiable.) More importantly, the sign and
monotonicities of a function may change depending on the sign of the input.
For instance, the {\tt square} function has positive sign (its outputs are
always positive) and is typically a nonmonotone function. However, when its
input is restricted to be positive, it is an increasing function. When a
\emph{convex} function has this property, we say that its monotonicity is 
\emph{signed}. This means it is increasing if the argument is positive, 
decreasing if the argument is negative, and nonmonotone otherwise. When a
\emph{concave} function has a signed monotonicity, it means that it is
decreasing if the argument is positive, increasing if the argument is negative,
and nonmonotone otherwise.

Before listing the atoms, a slight word on notation. We deviate slightly from
standard notation and use $f : \reals^p \to \reals^q$ to mean that a function
$f$ takes as input a real $p$-vector and produces a real $q$-vector as
output. Valid inputs are a \emph{subset} of all possible $p$-vectors,
specifically, its domain which we denote with $\dom f$. In other words, the
notation $f : \reals^p \to \reals^q$ can be thought of in the computer
science way---as a \emph{type signature} for the function $f$. As an example,
the $\log$ function has the following type signature, $\log : \reals \to
\reals$, although $\dom f$ is the set of strictly positive reals.

% We can take the analogy to computer science functions one step further: a function is overloaded and can have multiple type signatures. Its implementation depends on the matched type. For instance, the {\tt square} function can have the type signatures,
% \begin{align*}
% {\tt square} &: \reals \to \reals_+ \\
% {\tt square} &: \reals_+ \to \reals_+,
% \end{align*}
% where $\reals_+$ denotes the set of positive reals. The first {\tt square} function is nonmonotone; the second is increasing in its argument. The argument type (real expression or positive real expression) determines which
% version of the {\tt square} function to use.


% %\renewcommand\arraystretch{1.2}
% \BT
%   \centering
%   \begin{tabular}{|l|l|l|} \hline
%     & Atom (graph) & SOC canonical definition \\ \hline \hline
%     
%     \multicolumn{3}{|l|}{Functions with parameters}  \\ \hline
%     & $t = \phi^\mathrm{smult}(x;a)$ & $t = ax$  \\ \cline{2-3}
%     & $t = \phi^\mathrm{mmult}(x;A)$ & $t = Ax$ \\ \cline{2-3}
%     & $t = \phi^\mathrm{const}(;a)$ & $t = a$ \\ \hline \hline
%     
%     \multicolumn{3}{|l|}{Scalar functions, $\phi: \reals \to \reals$ (XXX: not quite right)} \\ \hline
%     & $t \geq \phi^\mathrm{pos}(x)$ & $t \in \soc^1, \; t-x \in \soc^1$ \\ \cline{2-3}
%     & $t \geq \phi^\mathrm{neg}(x)$ & $t \in \soc^1, \; t+x \in \soc^1$ \\ \cline{2-3}
%     & $t \leq \phi^\mathrm{geo\_mean}(x,y)$ & $\left((1/2)(y+x), (1/2)(y-x),t\right) \in \soc^{3}, \; y \in \soc^1$ \\ \cline{2-3}
%     & $t \geq \phi^\mathrm{square}(x)$ & $t \geq \phi^\mathrm{quad\_over\_lin}(x,1)$ \\ \cline{2-3}
%     & $t \leq \phi^\mathrm{sqrt}(x)$ & $t \leq \phi^\mathrm{geo\_mean}(x,1)$\\ \cline{2-3}
%     & $t \geq \phi^\mathrm{inv\_pos}(x)$ & $t \geq \phi^\mathrm{quad\_over\_lin}(1,x)$ \\ \cline{2-3}
%     & $t \geq \phi^\mathrm{abs}(x)$ & $(t,x) \in \soc^2$ \\ \cline{2-3}
%     & $t = \phi^\mathrm{plus}(x,y)$ & $t = x+y$ \\ \cline{2-3}
%     & $t = \phi^\mathrm{minus}(x,y)$ & $t = x-y$ \\ \cline{2-3}
%     & $t = \phi^\mathrm{negate}(x)$ & $t = -x$ \\ \hline \hline
%     
%     \multicolumn{3}{|l|}{Vector functions, $\phi: \reals^n \to \reals$ (XXX: not quite right)} \\ \hline
%     & $t \geq \phi^\mathrm{max}(x)$ & $t - x_i \in \soc^1, \, i=1,\ldots,n$ \\ \cline{2-3}
%     & $t \leq \phi^\mathrm{min}(x)$ & $x_i - t \in \soc^1, \, i=1,\ldots,n$ \\ \cline{2-3}
%     & $t \geq \phi^\mathrm{quad\_over\_lin}(x,y)$ & $\left((1/2)(y+t), (1/2)(y-t),x\right) \in \soc^{n+2}, \; y \in \soc^1$ \\ \cline{2-3}
%     & $t = \phi^\mathrm{sum}(x)$ & $t = \ones^Tx$ \\ \cline{2-3}
%     & $t \geq \phi^\mathrm{norm}(x)$ & $(t,x) \in \soc^{n+1}$ \\ \cline{2-3}
%     & $t \geq \phi^\mathrm{norm1}(x)$ & $t \geq \phi^\mathrm{sum}(\phi^\mathrm{abs}(x))$ \\ \cline{2-3}
%     & $t \geq \phi^\mathrm{norm\_inf}(x)$ & $t \geq \phi^\mathrm{max}(\phi^\mathrm{abs}(x))$ \\ \hline
%   \end{tabular}
%   \caption{List of atoms in DSL function library.}
%   \label{t-atoms}
% \ET
% * vector operators (map vectors to scalars)
%   * `sum(x)`
%   * `sum(x,y,..)`, defined as `x + y + ...`
% 
% The operators used for constructing second-order cones are:
% 
% * scalar operators (map scalars to scalars)
%   * `abs(x)`
% * vector operators (map vectors to scalars)
%   * `norm(x)`
%   * `norm2(x)`, equivalent to `norm(x)`
  
  
\subsubsection{Scalar atoms}
A scalar atom is a function with the type signature $f : \reals \to \reals$.
Table~\ref{t-scalar-atoms} lists out all the scalar atoms in our library
and their signs, monotonicity, and curvatures.
\begin{table}
  \small
  \renewcommand{\arraystretch}{1.5}
  \centering
\begin{tabular}{|l|l||l|l|l|} \hline
  Function & Definition & Sign & Monotonicity & Curvature \\ \hline
  {\tt abs(x)} & $f(x) = |x|$ & positive & signed & convex \\ \hline
  {\tt huber(x)} & $f(x) = \begin{cases} x^2 & |x| \leq 1 \\
    2|x| - 1 \leq 1 & |x| > 1 \end{cases}$ & positive & signed & convex \\ \hline
  {\tt inv\_pos(x)} & $f(x) = 1/x, \; x > 0$ & positive & decreasing & convex \\ \hline
  {\tt neg(x)} & $f(x) = \max(-x, 0)$ & positive & increasing & convex \\ \hline
  {\tt pos(x)} & $f(x) = \max(x, 0)$ & positive & increasing & convex \\ \hline
  {\tt pow\_rat(x,p,q)} & $f(x) = x^{p/q}, \; x >0$ & positive & (depends) & (depends) \\ \hline
  {\tt square\_over\_lin(x,y)} & $f(x) = x^2/y, \; y >0$ & positive & signed, decreasing & convex\\ \hline
  {\tt square(x)} & $f(x) = x^2$ & positive & signed & convex\\ \hline
  {\tt geo\_mean(x,y)} & $f(x,y) = \sqrt{xy}, \; x,y >0$ & positive & increasing, increasing & concave \\ \hline
  {\tt sqrt(x)} & $f(x) = \sqrt{x}, \; x >0$ & positive & increasing & concave \\ \hline
\end{tabular}
\caption{A list of the scalar atoms in \qcml. Scalar atoms with multiple arguments
require the same dimension for both arguments; their montonicities are listed separately.}
\label{t-scalar-atoms}
\end{table}

The {\tt pow\_rat} atom is slightly special in that it takes two \emph{integer
literal} arguments, {\tt p} and {\tt q}, which must be in the set $\{1,2,3,4\}$.
The monotonicity and curvature of the atom depend on the values of {\tt p}
and {\tt q} chosen. If $p > q$, the function is increasing and convex.
If $p < q$, the function is decreasing and concave. If $p = q$, the function
is affine and, moreover, is the identity function.

We can define the \emph{extension}, $\tilde f : \reals^n \to \reals^n$, 
of a scalar atom $f$ to a vector argument $x = (x_1, x_2, \ldots, x_n)$ as
\[
\tilde f(x) = \left( f(x_1), f(x_2), \ldots, f(x_n) \right),
\]
\ie, $\tilde f$ is $f$ applied elementwise.
Whenever scalar functions are called with vector arguments, this extension
is automatically applied. For scalar functions with the signature 
$f : \reals \times \reals \to \reals$, their extension has the signature
$\tilde f : \reals^n \times \reals^n \to \reals^n$.

\subsubsection{Vector atom}
A vector atom is a function with the type signature $f : \reals^n \to \reals$.
Table~\ref{t-vector-atoms} lists out all the scalar atoms in our library
and their signs, monotonicity, and curvatures.

\begin{table}
  \centering
  \small
  \renewcommand{\arraystretch}{1.5}
  \centering
\begin{tabular}{|l|l||l|l|l|} \hline
  Function & Definition & Sign & Monotonicity & Curvature \\ \hline
  {\tt quad\_over\_lin(x,y)} & $f(x) = x^Tx/y, \; y >0$ & positive & signed, decreasing & convex \\ \hline
  {\tt norm\_inf(x)} & $f(x) = \|x\|_\infty$ & positive & signed & convex \\ \hline
  {\tt norm1(x)} & $f(x) = \|x\|_1$ & positive & signed & convex \\ \hline
  {\tt norm(x)} or {\tt norm2(x)} & $f(x) = \|x\|_2$ & positive & signed & convex \\ \hline
  {\tt max(x)} & $f(x) = \max_i x_i$ & (depends) & increasing & convex \\ \hline
  {\tt min(x)} & $f(x) = \min_i x_i$ & (depends) & increasing & concave \\ \hline
\end{tabular}
\caption{A list of the vector atoms in \qcml. Vector atoms with multiple arguments
require the same dimension for both arguments; their montonicities are listed separately.}
\label{t-vector-atoms}
\end{table}

The {\tt quad\_over\_lin} atom is slightly special in that it has a mixed
type signature, $f : \reals^n \times \reals \to \reals$. Its second argument
is \emph{always} scalar. If a nonscalar is provided, an exception will be
raised.

The {\tt max} and {\tt min} atoms have signs which depend on the sign of their
arguments. The rule for the sign of {\tt max} is as follows:
\begin{itemize}
\item if \emph{any} of its arguments are positive {\tt max} is positive,
\item if \emph{all} its arguments are negative {\tt max} is negative,
\item otherwise, {\tt max} has unknown sign.
\end{itemize}
The rule for the sign of {\tt min} is as follows:
\begin{itemize}
\item if \emph{any} of its arguments are negative {\tt min} is negative,
\item if \emph{all} its arguments are positive {\tt min} is positive,
\item otherwise, {\tt min} has unknown sign.
\end{itemize}

With the exception of {\tt quad\_over\_lin}, vector atoms can take variable
arguments. The \emph{extension}, $\tilde f : \reals^n \times \reals^n \to
\reals^n$, of a vector atom $f$ to two vector arguments is
\[
\tilde f(x, y) = \left( f\left(\begin{bmatrix} x_1 \\ y_1 \end{bmatrix}\right), 
  f\left(\begin{bmatrix}x_2 \\ y_2 \end{bmatrix}\right), \ldots, 
  f\left(\begin{bmatrix}x_n \\ y_n\end{bmatrix}\right) \right).
\]
The extension to an arbitrary number of input arguments can be obtained inductively.
Whenever vector functions are called with variable arguments, this extension
is automatically applied.

As an example, the expression {\tt norm(x,y,z)} forms the vector expression
\[
\left(  
  \|(x_1,y_1,z_1)\|_2,
  \|(x_2, y_2, z_2)\|_2
\right),
\]
where $x = (x_1,x_2)$, $y = (y_1, y_2)$, and $z = (z_1, z_2)$. The extension
can be thought of as first performing a horizontal concatenation and applying
the atom \emph{row-wise}.

\subsection{Expressions}
Expressions in \qcml are formed by applying basic operators and functions to
numeric constants and to previously declared {\tt parameters} and {\tt
variables}. Every expression has additional properties:
\begin{itemize}
\item a \emph{shape} (either scalar, vector, or matrix),
\item a \emph{sign} (either positive, negative, or neither),
\item and a \emph{curvature} (either convex, concave, affine, or nonconvex).
\end{itemize}
An expression with \emph{nonconvex} curvature only means that it cannot be
verified as either convex, concave, or affine. The shape and sign of the
expression is inferred starting from the properties of the basic objects and
applying the operators and atoms (as listed in
Tables~\ref{t-scalar-atom,t-vector-atom}). The expression curvature is
inferred in a similar way but using disciplined convex programming (DCP)
rules. We detail the DCP rules in \S\ref{s-dcp}.

% For convenience in the sequel and with a slight abuse of notation, we use
% $\reals$ to refer to the set of all scalar-valued \emph{expressions} and $\reals_+$ to refer to the set of all scalar-valued expressions with positive sign. Similarly, $\reals^n$ and $\reals_-$ refer to the set of all vector-valued expressions of length $n$ and negative expressions, respectively.

%  associate shape scalar, vector, or matrix) and is categorized by its sign: positive, negative, or neither.
% 
% 
% .. blah blah, must adhere to DCP rules
% 
% can use numeric constants....
% 
% symbols must have been previously declared (basic types).
% 
% have sign and shape (vector or matrix) associated with them.


\subsection{Objectives}
Objectives are optional in \qcml. If no objective is provided, the problem
is assumed to be a feasibility problem.

\subsection{Constraints}
<=
>=
==

\subsection{Problems}
A \qcml problem consists of a list of basic type declarations, zero or one
objective, and zero or more constraints. Here is a \qcml problem for
completeness:
{\tt
\begin{tabbing}
\qquad 
\= dimensions m n \\
\> variable x(n) \\
\> parameter mu(n)\\ 
\> parameter gamma positive\\
\> parameter F(n,m) \\
\> parameter D(n,n) \\
\\
\> maximize (mu'*x - gamma*(square(norm(F'*x)) + square(norm(D*x)))) \\
\> subject to \\
\> \qquad \= sum(x) == 1 \\
\> \> x >= 0
\end{tabbing}
}
This sample problem has six type declarations, one maximization objective, and
two constraints.

\section{Disciplined convex programming}
\label{s-dcp}

\section{Abstract dimensions}
\label{s-abstract-dims}
Dimensions are initially specified as abstract values and can be left abstract
in \qcml. These abstract values must be converted into concrete values
before the problem can be solved. There are two ways to make dimensions 
concrete:
\begin{enumerate}
\item specified prior to code generation
\item specified after code generation by passing a {\tt dims} map (as a
  dictionary or struct) to the generated functions %, e.g. \texttt{prob2socp`, `socp2prob`
\end{enumerate}

Dimensions can be set prior to code generation by supplying
a partial map, \ie, if {\tt problem} contains a \qcml problem with 
dimensions \texttt{m} and \texttt{n}, then the Python code
{\tt
\begin{tabbing}
  \qquad \= >>> p = QCML() \\
  \> >>> p.parse(problem) \\
  \> >>> p.dims = \{'m': 5\}
\end{tabbing}
}
\noindent will set the dimension {\tt m} to $5$ in all expressions, but leave 
the dimension {\tt n} to be specified later.

Any dimensions specified before code generation will be hard-coded into the
resulting problem formulation functions. Thus all problem data fed into the
generated code must match these prespecified dimensions. In other words, the
generated code is \emph{specialized} to this fixed dimension.

Dimensions that are left abstract allow users to specify problems of
differing size, but the dimensions of the input problem must be supplied at
the same time. A future release may allow some dimensions to be inferred from
the size of the inputs.

\section{Code generation}
The 
\subsection{Python}
can view string
\subsection{Matlab}
can view string
\subsection{C}
Produces folder

\section{Rapid prototyping}
\qcml also provides a helper function for rapid prototyping in Python. This
is useful to verify the optimization model before generating code to use
elsewhere.

\section{Lasso example}
As an example, consider the lasso problem
\[
\begin{array}{ll}
  \mbox{minimize} & \|Ax - b\|_2^2 + \lambda \|x\|_1,
\end{array}
\]
with variable $x \in \reals^n$ and problem data $A\in\reals^{m\times n}$, 
$b \in \reals^m$, and $\lambda > 0$.

This problem can be specified as a \qcml problem as follows:
{\tt
\begin{tabbing}
\qquad 
\= dimensions m n \\
\> variable x(n) \\
\> parameters A(m,n) b(m)\\ 
\> parameter gamma positive\\
\\
\> minimize square(norm(A*x - b)) + gamma*norm1(x)
\end{tabbing}
}
\noindent blhbladslkasf

This example can be found under the {\tt qcml/examples} directory.

%     """
%     dimensions m n
% 
%     variable x(n)
%     parameter mu(n)
%     parameter gamma positive
%     parameter F(n,m)
%     parameter D(n,n)
%     maximize (mu'*x - gamma*(square(norm(F'*x)) + square(norm(D*x))))
%         sum(x) == 1
%         x >= 0
%     """
% 
% Our tool parses the problem and rewrites it, after which it can generate
% Python code or external source code. The basic workflow is as follows
% (assuming `s` stores a problem specification as above).
% 
%     p.parse(s)
%     p.canonicalize()
%     p.dims = {'m': 5}
%     p.codegen('python')
%     socp_data = p.prob2socp({'mu':mu, 'gamma':1,'F':F,'D':D}, {'n': 10})
%     sol = ecos.solve(**socp_data)
%     my_vars = p.socp2prob(sol['x'], {'n': 10})
% 
% We will walk through each line:
% 
% 1. `parse` the optimization problem and check that it is convex
% 2. `canonicalize` the problem by symbolically converting it to a second-order
%    cone program
% 3. assign some `dims` of the problem; others can be left abstract (see [below] (#abstract-dimensions))
% 4. generate `python` code for converting parameters into SOCP data and for 
%    converting the SOCP solution into the problem variables
% 5. run the `prob2socp` conversion code on an instance of problem data, pulling 
%    in local variables such as `mu`, `F`, `D`; because only one dimension was 
%    specified in the codegen step (3), the other dimension must be supplied when 
%    the conversion code is run
% 6. call the solver `ecos` with the SOCP data structure
% 7. recover the original solution with the generated `socp2prob` function; 
%    again, the dimension left unspecified at codegen step (3) must be given here
% 
% For rapid prototyping, we provide the convenience function:
% 
%     solution = p.solve()
% 
% This functions wraps all six steps above into a single call and
% assumes that all parameters and dimensions are defined in the local
% namespace.
% 
% Finally, you can call
% 
%     p.codegen("C", name="myprob")
%   
% which will produce a directory called `myprob` with five files:
% 
% * `myprob.h` -- header file for the `prob2socp` and `socp2prob` functions
% * `myprob.c` -- source code / implementation of the two functions
% * `qc_utils.h` -- defines static matrices and basic data structures
% * `qc_utils.c` -- source code for matrices and data structures
% * `Makefile` -- sample Makefile to compile the `.o` files
% 
% You can include the header and source files with any project, although you must
% supply your own solver. The code simply stuffs the matrices for you; you are
% still responsible for using the proper solver and linking it. An example of how
% this might work is in `examples/lasso.py`.
% 
% The `qc_utils` files are static; meaning, if you have multiple sources you wish 
% to use in a project, you only need one copy of `qc_utils.h` and `qc_utils.c`.
% 
% The generated code uses portions of CSparse, which is LGPL. Although QCML is 
% BSD, the generated code is LGPL for this reason.
% 
% For more information, see the [features](#features) section.
% woot
\newpage
\bibliographystyle{plain}
\bibliography{userguide}
\end{document}