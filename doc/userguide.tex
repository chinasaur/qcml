\documentclass[11pt]{article}

\usepackage{fullpage, xspace, hyperref, amsmath}

\input defs.tex

\def\qcmlver{0.1\xspace}
\def\qcml{\texttt{QCML}\xspace}
\def\cvxpy{\texttt{CVXPY}\xspace}
\def\cvx{\texttt{CVX}\xspace}
\def\cvxgen{\texttt{CVXGEN}\xspace}


\title{\qcml version \qcmlver \\
A Python Toolbox for Matrix Stuffing}

\author{
Eric Chu\\\texttt{echu508@stanford.edu}
\and
Stephen Boyd\\\texttt{boyd@stanford.edu}
}

\date{\today}

\begin{document}
  \maketitle
  
\qcml is a Python module that provides a simple and lightweight domain
specific language (DSL) for describing convex optimization problem families
representable as second-order cone problems (SOCP). It is a code generator
that produces lightweight Python, Matlab, or C code for \emph{matrix
stuffing}. This allows one to extend SOCP solvers in these langauges to a 
variety of problem classes. It is intended to complement the paper, 
\emph{Code Generation for Embedded Second-order Cone Programming} \cite{ECC}, 
although the main ideas in its
implementation are well-documented in \emph{Graph Implementation}
\cite{Grant}.

\qcml comes with a small example library and consists of a single Python
object, \texttt{QCML}, with three methods
\BIT
\item \texttt{parse}, which parses an optimization problem specified as a
string and checks that it is convex,
\item \texttt{canonicalize}, which symbolically converts the problem into
a second-order cone program, and
\item \texttt{codegen}, which produces source code that performs the
transformation from problem-specific data to second-order cone problem data
and back.
\EIT
The \texttt{codegen} method can be used to generate Python, C, and Matlab
source code.

Some important features:
\BIT
\item \qcml is similar in spirit to \cvx, \cvxgen, and 
\cvxpy. Unlike these packages, however, \qcml exposes and generates code
for matrix stuffing, allowing the resulting code to be paired with any solver
that targets a standard SOCP.
\item \qcml is capable of generating fully abstract code with problem data and
dimensions as input to the \emph{generated} code, although it can specialize
the code if given problem dimensions before code is generated.
\item \qcml supports matrix parameters with arbitrary sparsity patterns.
\EIT
That said, there are several important caveats:
\BIT
\item \qcml is intended for use by \emph{developers} wishing to employ convex
optimization in their packages with little to no overhead from the modeling 
layer. Typically, these are users familiar with \cvx or \cvxpy who wish to
do away with modeling overhead in their applications.
\item \qcml is not a full-featured modeling language. Several features are
conspicuously missing and unless specifically requested, will remain missing.
\item For general purpose convex optimization in Python with more flexibility
than \qcml, we recommend \cvxpy 
(\url{http://github.com/cvxgrp/cvxpy}).
\item For general purpose convex optimization in Matlab, we recommend 
\cvx (\url{http://cvxr.com}).
\EIT
Please send your feedback or report any bugs through our Github repository.

The rest of this document covers the basic use of \qcml and walks through a
full example with a lasso ($\ell_1$-regularized least-squares) problem. 
\qcml is licensed under BSD. 

\section{Installing \qcml}
\qcml depends on the following:
\BIT 
\item Python 2.7.2+ (no Python 3 support)
\item \href{http://www.dabeaz.com/ply/}{\tt PLY}, the Python Lex-Yacc parsing framework.
  Available as the {\tt python-ply} or {\tt py-ply} package in most distributions.
\item {\tt ECOS} (\url{http://github.com/ifa-ethz/ecos})
\item {\tt NUMPY} (\url{http://numpy.org})
\item {\tt SCIPY} (\url{http://scipy.org})
\EIT
For (some) unit testing, we use Nose (\url{http://nose.readthedocs.org}).

Once these dependencies are installed, typically through a package manager,
\qcml can be installed by typing
{\tt
\begin{tabbing}
  \qquad \= \$ python setup.py install
\end{tabbing}
}
\noindent at the command line. The unit tests can be run with {\tt nosetests}.

\section{\qcml language}
Optimization problems in \qcml are specified as strings which must adhere to 
the rules of the \qcml modeling language. We describe that language and its
rules here.

\subsection{Basic types}
There are three basic types in \qcml:
\BIT
\item {\tt dimension} (or {\tt dimensions} for multiple dimension declarations)
\item {\tt parameter} (or {\tt parameters} for multiple parameter declrations)
\item {\tt variable} (or {\tt variables} for multiple variable declarations)
\EIT
Objects named and declared in this fashion are entirely abstract.

The {\tt dimension} objects are used to specify the (optinal) sizes of {\tt
variable}s and {\tt parameter}s. So a single {\tt variable} or {\tt
parameter} object can represent an collection (or array) of scalar variables
or parameters with abstract dimension. (For more detail on abstract
dimensions, consult \S\ref{s-abstract-dims}.) At the moment {\tt variables}
can represent vectors (one-dimensional arrays), and {\tt parameters} can
represent vectors or matrices (two-dimensional arrays). If a {\tt parameter}
represents a matrix, it is assumed to represent a \emph{sparse} matrix.

A {\tt parameter} may optionally take a sign argument ({\tt positive} or 
{\tt negative}). These are shorthand `nonnegative' and `nonpositve', 
respectively.

To declare an object, give its type, followed by its name, and (optionally) 
its size:
{\tt
\begin{tabbing}
  \qquad \= variable x(n)
\end{tabbing}}
\noindent declares a vector variable with dimension {\tt n} and named {\tt x}.
A more involved example is the following code
{\tt
\begin{tabbing}
  \qquad \= dimensions m n \\
  \> variables x(n) y(m) \\
  \> parameter A(m,n) positive \\
  \> parameters b c(n)
\end{tabbing}}
\noindent which declares two dimensions, {\tt m} and {\tt n}; two variables
{\tt x} and {\tt y} of length {\tt n} and {\tt m}, respectively; an
elementwise positive (sparse) parameter matrix, {\tt A}; the scalar parameter
{\tt b}; and the vector parameter {\tt c} with dimension {\tt n}.

\subsection{Basic operations and functions}
QCML provides a set of linear operators and atoms for use with modeling.
Since an SOCP only consists of affine functions and second-order cone
inequalities, we only provide linear operators and operators for constructing
second-order cones. All other atoms are implemented as *macros*. Whenever the
parser encounters an atom, it simply expands its definition.

Operators
---------
The standard linear operators are:

* infix operators
  * `+`
  * `-`
  * `*`, lhs *must* be a parameter
  * `/`, rhs and lhs *must* be numeric constants
* prefix operators
  * `-`, unary minus / negate
* vector operators (map vectors to scalars)
  * `sum(x)`
  * `sum(x,y,..)`, defined as `x + y + ...`

The operators used for constructing second-order cones are:

* scalar operators (map scalars to scalars)
  * `abs(x)`
* vector operators (map vectors to scalars)
  * `norm(x)`
  * `norm2(x)`, equivalent to `norm(x)`
  
  
\subsubsection{Scalar atoms}
A scalar atom is a function with the \emph{type signature} 
$f : \reals \to \reals$. This means the function $f$ takes as input scalar
real numbers and produces a scalar real number as output. Note that this 
differs slightly from standard notation, in which the domain is explicit; 
our notation means that $f$ is
scalar-valued function over a subset of $\reals$, specifically, its domain,
which we denote with $\dom f$. As an example, the $\log$ function has the
following type signature, $\log : \reals \to \reals$, although $\dom f$ is
the set of strictly positive reals.

We can define the \emph{extension}, $\tilde f : \reals^n \to \reals^n$, 
of a scalar atom $f$ to vector arguments as
\[
\tilde f(x) = \left( f(x_1), f(x_2), \ldots, f(x_n) \right).
\]
Whenever scalar functions are called with vector arguments, this extension
is automatically applied.

* scalar atoms (map scalars to scalars)
  * `pos(x)`, defined as `max(x, 0)`
  * `neg(x)`, defined as `max(-x, 0)`
  * `square(x)`
  * `inv\_pos(x)`
  * `geo\_mean(x,y)`
  * `pow\_rat(x,p,q)', defined as `x\^(p/q)' only for 1 <= p,q <= 4
  * `sqrt(x)`

\subsubsection{Vector atom}
A vector atom is a function with the \emph{type signature} 
$f : \reals^n \to \reals$.

Vector atoms can take variable arguments. 
The \emph{extension}, $\tilde f : \reals^n \times \reals^n \to \reals^n$, of 
a vector atom $f$ to two arguments is
\[
\tilde f(x, y) = \left( f\left(\begin{bmatrix} x_1 \\ y_1 \end{bmatrix}\right), 
  f\left(\begin{bmatrix}x_2 \\ y_2 \end{bmatrix}\right), \ldots, 
  f\left(\begin{bmatrix}x_n \\ y_n\end{bmatrix}\right) \right).
\]
The extension to an arbitrary number of input arguments can be obtained inductively.
Whenever vector functions are called with variable arguments, this extension
is automatically applied.

* vector atoms (map vectors to scalars)
  * `max(x)`, the max elem of `x`
  * `max(x,y,..)`, the max vector consisting of max elements across rows
  * `min(x)`, the min elem of `x`,
  * `min(x,y,..)`, the min vector consisting of min elements across rows
  * `quad\_over\_lin(x,y)`, if `y` is a vector, returns element-wise operator
  * `norm1(x)`, defined as `sum(abs(x))`
  * `norm1(x,y,..)`, defined as `abs(x) + abs(y) + ...`
  * `norm\_inf(x)`, defined as `max(abs(x))`
  * `norm\_inf(x,y,...)`, defined as `max(abs(x),abs(y),...)`

\subsection{Expressions}
Expressions in \qcml.. blah blah, must adhere to DCP rules

can use numeric constants....

symbols must have been previously declared (basic types).

\subsection{Objectives}
\subsection{Constraints}
\subsection{Problems}
A \qcml problem consists of a list of basic type declarations, zero or one
objective, and zero or more constraints. 


\section{Abstract dimensions}
\label{s-abstract-dims}
Dimensions are initially specified as abstract values and can be left abstract
in \qcml. These abstract values must be converted into concrete values
before the problem can be solved. There are two ways to make dimensions 
concrete:
\begin{enumerate}
\item specified prior to code generation
\item specified after code generation by passing a {\tt dims} map (as a
  dictionary or struct) to the generated functions %, e.g. \texttt{prob2socp`, `socp2prob`
\end{enumerate}

Dimensions can be set prior to code generation by supplying
a partial map, \ie, if {\tt problem} contains a \qcml problem with 
dimensions \texttt{m} and \texttt{n}, then the Python code
{\tt
\begin{tabbing}
  \qquad \= >>> p = QCML() \\
  \> >>> p.parse(problem) \\
  \> >>> p.dims = \{'m': 5\}
\end{tabbing}
}
\noindent will set the dimension {\tt m} to $5$ in all expressions, but leave 
the dimension {\tt n} to be specified later.

Any dimensions specified before code generation will be hard-coded into the
resulting problem formulation functions. Thus all problem data fed into the
generated code must match these prespecified dimensions. In other words, the
generated code is \emph{specialized} to this fixed dimension.

Dimensions that are left abstract allow users to specify problems of
differing size, but the dimensions of the input problem must be supplied at
the same time. A future release may allow some dimensions to be inferred from
the size of the inputs.

\section{Code generation}
The 
\subsection{Python}
can view string
\subsection{Matlab}
can view string
\subsection{C}
Produces folder

\section{Rapid prototyping}
\qcml also provides a helper function for rapid prototyping in Python. This
is useful to verify the optimization model before generating code to use
elsewhere.

\section{Lasso example}
As an example, consider the lasso problem
\[
\begin{array}{ll}
  \mbox{minimize} & \|Ax - b\|_2^2 + \lambda \|x\|_1,
\end{array}
\]
with variable $x \in \reals^n$ and problem data $A\in\reals^{m\times n}$, 
$b \in \reals^m$, and $\lambda > 0$.

This problem can be specified as a \qcml problem as follows:
{\tt
\begin{tabbing}
\qquad 
\= dimensions m n \\
\> variable x(n) \\
\> parameters A(m,n) b(m)\\ 
\> parameter gamma positive\\
\\
\> minimize square(norm(A*x - b)) + gamma*norm1(x)
\end{tabbing}
}
\noindent blhbladslkasf

This example can be found under the {\tt qcml/examples} directory.

%     """
%     dimensions m n
% 
%     variable x(n)
%     parameter mu(n)
%     parameter gamma positive
%     parameter F(n,m)
%     parameter D(n,n)
%     maximize (mu'*x - gamma*(square(norm(F'*x)) + square(norm(D*x))))
%         sum(x) == 1
%         x >= 0
%     """
% 
% Our tool parses the problem and rewrites it, after which it can generate
% Python code or external source code. The basic workflow is as follows
% (assuming `s` stores a problem specification as above).
% 
%     p.parse(s)
%     p.canonicalize()
%     p.dims = {'m': 5}
%     p.codegen('python')
%     socp_data = p.prob2socp({'mu':mu, 'gamma':1,'F':F,'D':D}, {'n': 10})
%     sol = ecos.solve(**socp_data)
%     my_vars = p.socp2prob(sol['x'], {'n': 10})
% 
% We will walk through each line:
% 
% 1. `parse` the optimization problem and check that it is convex
% 2. `canonicalize` the problem by symbolically converting it to a second-order
%    cone program
% 3. assign some `dims` of the problem; others can be left abstract (see [below] (#abstract-dimensions))
% 4. generate `python` code for converting parameters into SOCP data and for 
%    converting the SOCP solution into the problem variables
% 5. run the `prob2socp` conversion code on an instance of problem data, pulling 
%    in local variables such as `mu`, `F`, `D`; because only one dimension was 
%    specified in the codegen step (3), the other dimension must be supplied when 
%    the conversion code is run
% 6. call the solver `ecos` with the SOCP data structure
% 7. recover the original solution with the generated `socp2prob` function; 
%    again, the dimension left unspecified at codegen step (3) must be given here
% 
% For rapid prototyping, we provide the convenience function:
% 
%     solution = p.solve()
% 
% This functions wraps all six steps above into a single call and
% assumes that all parameters and dimensions are defined in the local
% namespace.
% 
% Finally, you can call
% 
%     p.codegen("C", name="myprob")
%   
% which will produce a directory called `myprob` with five files:
% 
% * `myprob.h` -- header file for the `prob2socp` and `socp2prob` functions
% * `myprob.c` -- source code / implementation of the two functions
% * `qc_utils.h` -- defines static matrices and basic data structures
% * `qc_utils.c` -- source code for matrices and data structures
% * `Makefile` -- sample Makefile to compile the `.o` files
% 
% You can include the header and source files with any project, although you must
% supply your own solver. The code simply stuffs the matrices for you; you are
% still responsible for using the proper solver and linking it. An example of how
% this might work is in `examples/lasso.py`.
% 
% The `qc_utils` files are static; meaning, if you have multiple sources you wish 
% to use in a project, you only need one copy of `qc_utils.h` and `qc_utils.c`.
% 
% The generated code uses portions of CSparse, which is LGPL. Although QCML is 
% BSD, the generated code is LGPL for this reason.
% 
% For more information, see the [features](#features) section.
% woot
\end{document}